\LARGE{ \textbf {Лекция №7}}\\
\Large{ \textbf {Двухфазный симплекс метод.}}\\
Второй подход в линейном программировании искуственного базиса называется двухфазный симплекс метотод.
В нем процесс вычисления разделен на 2 этапа, поэтому такое название.
На первом этапе нужно найти опорный допустимый базис.
На втором этапе нужно его оптимизировать.

На первом этапе рассматривается искуственная задача с искуственными переменными в ограничениях, а целевая функция в ней
есть сумма $z' = \sum\limits_i{R_i} $ ищем минимум этой суммы вне зависимости от того, что нужно найти в исодной задаче.

Поэтому в канонической форме такая искуственная функция цели записывается так:\\
$-z' +  \sum\limits_i{R_i} = 0 $
Искуственные переменные R образуют опорный искуственный базис, поэтому они должны входить в z с кэфом 0, а они входят с кэфом 1.

Надо исключить из функции цели
Выразить кажду иск переменную из своего какно уравнения и поставить в функцию цели.

$
\sum\limits_{j=1}^n{a_ij x_j} - x_n+1 + R_i = b_i
$

$
R_i = b_i +  x_n+1 (- \sum\limits_{j=1}^n{a_ij x_j} )
$

В этом случае система ограничений юудут равны. Это значит, что решение полученное на первом этапе можно рассматривать,как опорный базис для последующей оптимизации.

Может поучится так, что оптимальное решение вспомогательной задачи оказалось больше 0, отсюда следует неравенство нулю некторох $R_i$, в результате нельзя говорить об эквивалентности этих задач.
При такой ситуации делается вывод: нет решений. Второй этап в этом случае не проводится.

Достигунуто $z = 0$\\
В этом случае нужно перейти ко второму этапу.

На втором этапе оптимальное решений этапа 1 рассматривается как опорный базис в исходной задаче.
При этом оптимальная симплекс таблица этапа 1 будет исходной таблицей для этапа 2.
ПРи этом из нее надо вычеркнуть столбцы искуственным перерменных $R_i$. \\
И во-вторых нужно заменить строку z искуственной функции цели на строку z исходнуй функции цели.

При этом однако оказывается, что в исходную целевую функцию входят некоторые переменные опорного базиса.
Нужно исключить их из целевой цункции.

Предположим: $x^B_q > 0$

Для исключения этой переменной $x_q$ из целвой функции нужно умножить базисное уравнение для переменной  $x_q$ на ее коэфицент целевой функции.\\
$x_q + \sum\limits_{j}{a_ij x_j} = b_q \cdot (-c_q)$

Те же действия удобно сделать по симплекс таблице
строку базисной переменной Надо умножить на коэфицент перед ней и вычесть из строки целевой функции. Результат записать вместо строки Z.
Такое действие нужно сделать для каждой базисной переменной .
После этого решение задачи производится прямым алгоритмом симплекс-метода.
И оптимальный результат этапа 2 будет оптимальным решением исходной задачи.
