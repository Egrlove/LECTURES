\documentclass{report}
\usepackage[left=2cm,right=2cm,top=2cm,bottom=2cm,bindingoffset=0cm]{geometry}
\usepackage[utf8x]{inputenc}
\usepackage[english,russian]{babel}
\usepackage{amsfonts}
\usepackage{amsmath}
\usepackage{mathtools}

\usepackage{graphicx}
\graphicspath{{pictures/}}
\DeclareGraphicsExtensions{.pdf,.png,.jpg}

\begin{document}
\LARGE{ \textbf {Лекция №4}}\\
\Large{ \textbf {}}\\

int socket(int dom, int type,int proto)\\
1 - коммуникациооная область: AF_UNIX or AF_INET\\
2 - тип протокола пердачи данных SOCK_STREAM , SOCK_DGRAM, SOCJ_RAW.\\
3 - протокол для сокета указнного уровня. у нас всегда 0.\\
Возврат - id сокета,полный аналог файлового дескриптора.\\

Клиент -Сервер.\\
Клиент:\\
s = socket()\\
[bind(s,..)]\\
connect(s, ..)\\
recv(s,..)\\
write(s,..)\\
shutdown(s, 2)\\
close(s) \\

Сервер: \\
s = socket()\\
bind(s,...)\\
listen(s,..)\\
snew = accept(s,...)\\
send(snew,..)\\
read(snew,..)\\
shutdown(snew, 2)\\
close(snew,..)\\



Для начала и тот и другой должный создать сокет. Его нужно поименовать.\\
int bind(int socket, name , int length )\\
name - указатель на структуру.\\
struct sockadd2{un in} *name;\\
un - имя файла в опер. системе. Unix Domain.  \\
in - internet domain, ip *2 , port * 2, протокол. \\

int listen(int s, int n) - организовать очередь длиной n.\\
int connect(int socket, name , int length ) - соединиться черз сокет сб с сервером из второго аргумента, длина адреса.\\
int accept(int socket, name , int *length ) - принять запрос на соед. будет размещен в область name адрес клиента.
В случае удачи возвращает дескриптор сокетаб такой же как первый, только дальнейший обмен должен вестись через него.\\

Windows:\\
int send(int s, char* buf, int n, int flags) buf -что посылаем, n -размер buf \\
int recv(int s, char* buf, int n, int flags)\\
flags:\\
MSG_OOB - out of band, экстренные сообщения\\
MSG_PEEK - при приеме данных, прочетать данные из сокета, оставив их в сокете\\


При datagram сокете есть специальные функции.
int send(int s, char* buf, int n, int flags, name, int len) name - тот же самый адрес \\
int recv(int s, char* buf, int n, int flags, name, int *plen)\\


shutdown(int s, int how)\\
0 - сброс на чтение\\
1 - сброс на запись, не пытаться отослать.\\
2 - и то , и то\\


Пример:\\
Сервер:\\
#define S_NAME "/tmp/s.sock"
#define B_SIZE 256
main(){
int socket, new_socket, len;
char buf[B_SIZE];

struct sock_addr_un sname, fname;
s = socket(AF_UNIX, SOCK_STREAM, 0);
sname.sun_family = AF_UNIX;
strcp(sname.sun_path, S_NAME);
bind(s, &sname, sizeof(sname));
listen(s,5);
while(1)
{
len = sizeof(fname);
snew= accept(s,&fname, &len);
send(snew, "Who are you?", 13 , 0);
read(snew,buf, B_SIZE);
shutdown(snew,2);
close(snew);
}
}

Клиент:\\
#define S_NAME "/tmp/s.sock"
#define B_SIZE 256
#define C_NAME "/tmp/c.sock"
main(){

int s, len;
char buf[B_SIZE];

struct sock_addr_un sname, cname;
s = socket(AF_UNIX, SOCK_STREAM, 0);
cname.sun_family = sname.sun_family = AF_UNIX;
strcp(cname.sun_path, C_NAME);
strcp(sname.sun_path, S_NAME);

s= socket(AF_UNIX, SOCK_STREAM,0);
bind(s, &cname, sizeof(cname));
connect(s, &sname, sizeof(sname));
recv(s, buf,B_SIZE);
write(s, "I am ",9);
shutdown(s, 2);
close(s);
ulink(C_NAME);



}


\end{document}
