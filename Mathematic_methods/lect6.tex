\LARGE{ \textbf {Лекция №6}}\\
\Large{ \textbf {Неограниченные оптимальные решения.}}\\

В этом случае отсутсвуют какие-либо условия, которые ограничивают изменения целевой функции в направлении оптимизации.
В результате Z может быть бесконечно велика или мала.
$ max Z - infiniti + $
В этом случае область допусттимых решений у такой задачи неограничена в направлении градиента или антиградиента.

Получается, что если это задача нахождения максимума, то можно бесконечно увеличивать значение.
Важное замечение - что неограничена она должна быть в направлении оптимизации.

\textbf {В итерациях симплекс-метода признаком неограниченных оптимальных решений является отсутвие положительных коэфицентов в ведущем столбце.}

Вырожденое решение.

Таким решением называется базисное решение, где равны 0 не только свободные, но и хотя бы одна базисная.
Вырожденными может быть как промежуточное так и оптимальное.
Получение вырожденного решения не придит к практическим неудобствам, кроме некоторых специальных задач,
где происходит зацикливание вычислительного процесса симплекс-метода.
Зацикливание означает, повторение одного и того же базисного решения, без продвижения по экстремому.
Однако обычно такое не происходит, но может быть так,
что на 2 соседних итерациях значение целевой функции оказывается одинаковым при различных вырожденных базисах.
Но базис у этих итераций будеи различный.
Несмотря на все теоритические неудобства нужно продолжать итерации симплекс-метода до получения оптимального решения.

В некоторых случаях оказывается полезным изменить правила выбора ведущего столбца.
Выбрать столбец с отрицательным коэфом в строке Z, который не является минимальным.

В плоском геометрическом случае вырожденным решениям соответсвует пересечению более, чем 2 границ в одной из его вершин.

Целеноправленность итераций и допустимость базиса.\\
Важной особенность прямого алгоритма симплекс-метода является то, что его итерации должны начинаться с опорного допустимого базиса.
А на всех остальных итерациях полученный базис гарантировано является допустимым.

Поскольку количество базисно допустимых решений конечно, то алгоритм симлекс-метода приходит к оптимальному решению за конечное число шагов.
При этом на каждой следующей итерации значение целевой функции гарантировано не хуже, чем на предыдущей итерации.
Это означает целеноправленность алгоритма и приводит к направленноиу перебору базисных решений вдоль поиска оптимума.


\Large{ \textbf {Линейное программирование в искуственном базисе.}}

Для реализации итерации прямого алгоритма симплекс метода необходим выбор допустимого опорнога базиса.
Его автоматически образуют остаточные переменные, если ограничения задачи это неравенства типа меньше или равно.
Если это не так, то есть есть типа больше или равно, или есть ограничения типа равенство, то возникает проблема с выбором базисных перменных,
в частности из уравнений какнонической формы с избыточными перменными.
Или из уравнений канонической формы, без дополнительных переменных.

Если в подстановке задачи есть ограничения больше иои равно, то нужно ввести в левую часть избыточную переменную со знаком '-'.\\
$ a_ij x_j => b_j $, то $ a_ij x_j - x_(n+j )=> b_j $ \\

Рациональный подход к решению это проблемы, состоит в том ,
чтобы ввести в левую часть каждого уравнения канонической формы с избыточной переменной или без допустимых переменных
дополнительную неотрицательную переменную.

$ a_ij x_j^F - x_(n+j )^F + R_i^B = b_i $
R - искуственная.
В этой задаче опорный базис образуют остаточные и искуственные переменные. Свободными являются исходные переменные и избыточными.

Введение искуственных переменных может быть оправдано только в том случае,
если предложить для их обработки такую вычислительную процедеру,
которая будет гарантировать обнуление искуственных переменных в оптимальном решении искуственной задачи.
В этом случае исходная и искуственная задача будут идентичны.

Имеется по крайней мере 2 процедуры обработки искуственного базиса, которые гарантируют разные эффекты.
Это метод больших штрафов (М-метод) и двухфазный симплекс-метод.
Их вычислительные схемы используют прямой алгоритм симплекс-метода.

В методе больших штрафов рассматривается искуственная штрафная задача, вместо исходной.
Ее образуют искуственные исходные ограничения, а вместо иходной целевой функции в ней рассматриваются штрафная функция цели.

$z = \sum{n}{j} c_j x_j \\
z' = \sum{n}{j} c_j x_j (+-) MR_i \\
$
М - стремитсся к бескону.
+ если миниммум.
- если максимум.

Если в ходе симплексной обработки штрафной задачи получилось оптимальное решение с нулевыми искуственными переменными,
то это означает, что штрафная задача идентична исходной.
Но если получается, что в оптимальном решении есть ненулевые искуственные переменнык,
то это означает, что исходная задача вообще не имеет допусимых решений.

Для практической реализации такого подхода нужно перед началом симплексных итераций исключить из
штрафной функции цели базисные искуственные переменные,
потому что в симплекс-методе в целевой функции не должно быть базисных переменных.

Исключение можно осуществить выразив каждую искуственную переменную из её уравнения.
Вычисления на итерациях нужно проводить в алгебраической форме относительно штрафного коэфицента. Это значит, что вместо него нельзя подставлять числовую константу.
Но можно рассматривать, как алгебраическую константу бесконечно большую.
