\documentclass{report}
\usepackage[left=2cm,right=2cm,top=2cm,bottom=2cm,bindingoffset=0cm]{geometry}
\usepackage[utf8x]{inputenc}
\usepackage[english,russian]{babel}
\usepackage{amsfonts}
\usepackage{amsmath}
\usepackage{mathtools}

\usepackage{graphicx}
\graphicspath{{pictures/}}
\DeclareGraphicsExtensions{.pdf,.png,.jpg}

\begin{document}
\LARGE{ \textbf {Лекция № 6}}\\
\Large{ \textbf { Последовательсные схемы.
Триггерные схемы.}} \\

Структорной ячейкой в комбинационных схемах является логический элемент.\\
В основе последовательсных схем (схем с памятью) лежит триггер.\\
Триггер - устройство с 2 устойчивыми состояниями.\\
Пример электромеханического триггера является выключатель.\\

Обобщенная структурная схема триггера.\\
$x_1, x_n ;$ - информационные входы , $c_1, c_n$ - входы синхронизации\\
$Q$- прямой выход. $\overline{Q}$ - обратный выход.\\
Состяние триггера определяется его выходами.\\
Q = 0  $\overline{Q}$ = 1 триггер выключен.\\
Q = 1  $\overline{Q}$ = 0 триггер вкючен.\\
S` , R`  - сигналы, которые действуют на входы запоминающей ячейки.\\
S` - установка в 1.\\
R` - установка в 0.\\
Запоминающая ячейка - простейшая схема триггера.\\


\Large{ \textbf { Классификация Триггеов }} \\
ПО функциональному призанаку различают:
\begin{enumerate}
  \item RS
  \item JK
  \item DV(E)
  \item T
  \item RSD, JKRS, DRS - комбинировнные.
\end{enumerate}
$5^(2^n)$ - количество различных типов триггеров

ПО способу записи информации\\
\begin{enumerate}
  \item Асинхронные (несинхронизируемые)
  \item Синхронные (тактируемые)
\end{enumerate}

По числу входов синхронизации
\begin{enumerate}
  \item Однотактные
  \item Многотактные
\end{enumerate}

По способу организации синхронизируещего выхода\\

\begin{enumerate}
  \item Синхронные триггеры со статическим управлением записью. (Запись происходит по уровню сигнала, горизонтальная часть)
  \item Синхронные триггеры с динамическим управлением записью. (По фронту, вертикальная часть)
\end{enumerate}

По способу передачи входно информации на выход триггера.\\
\begin{enumerate}
  \item Одноступеньчатые
  \item Двухступеньчатые
\end{enumerate}

Основые параметры триггеров.\\
Схемотехнические параметры.\\

Такие же как у логических элементов.\\
Справочники по микросхемам.\\

Кэф объединения по входу, разъединения по выходу. $К_об  \quad К_раз$\\
Характеризуют нагрузочную способность.\\


Напряжение допустимой помехи. \\
Входные и выходные напряжения и токи логического 0 и 1. \\

Потребляемая мощность.\\


Динамические параметры.\\

Минимальная длительность входного сигнала. $t_u$ \\
Длительность при которой еще происходит переключение триггера.\\
$ t_u = \sum{n}{i = 1} t^* \cdot n  $
$t^*$ - среднее время задержки распространения сигнала  в лог элементе.

$n$ - колво логических элементов от входа  (синхросигнала) до входа логического элемента на котором замыкается кольцо обратной связи.

$t_p$ - разрешающее время триггера, минимально допустимый временной интервал между 2 последовательными входными сигналами
 минимальной длительности вызывыющими переключение триггера.

$f_max = \frac{1}{t_p}$ - Максимальная частота переключения триггера , обратна разрешающему времени.
$f_rab = \frac{f_max}{1,5}$  - указывается в технических условиях для наихудшего случая.

$t^tg = \sum{m}{i = 1} t^* $  Среднее время задержки распространения сигнала в триггере.
$m = n + 1$ - колво логических элементов цепи от входа информационного или синхросигнала, до выхода логического элемента на котором подтверждается расстояние триггера.



Область применения триггеров
\begin{itemize}
  \item Счетчики
  \item Маленьких ОЗУ
  \item Накапливающих сумматорах SM
  \item Генераторах одниночных импульсов
  \item В формирователях сигналов
\end{itemize}
Используются там, где необходимо хранение информации.\\

\Large {Логическое описание работы триггеров.}

Правило функционирования триггеров может быть задано:
\begin{itemize}
  \item Словесным описанием (Вербальным описанием)
  \item С помощью логических функций
  \item В виде таблицы переходов
  \item С помощью графов, в котором вершинам соответсвует внутреннее состояние триггера, ветвям - управляющий сигнал.
  \item В форме микропрограммного автомата.
\end{itemize}

При описании работы триггера рассматривают 2 соседних момента времени $t_n \quad t_(n+1)$\\
Для асинхронных триггеров $t_n+1$ наступает, когда под действием входных сигналов
и в зависимости от внутреннего состояния в момент $t_n$ триггер переходит в последующее состояние\\

В синхронных триггерах $t_n$ и $t_n+1$ время до и после прихода синхросигнала соответсвенною

Асинхронный RS триггер.

По опредению :
R -reset установка в 0 \\
S - set  установка в 1 \\

Используется в качестве запоминающей ячейки в более сложных схемах.\\
Задача построить характерестическую логическую функцию.\\
$Q_n+1 = f(R_n , S_n, Q_n)$
Для прямого  3 выражение\\
$Q_n+1 = S_n + \overline{R_n} Q_n$
$Q_n+1 = \overline {S_n + \overline{R_n + \overline{Q_n}}} $\\

Для инверсного:  4  выражение\\
$Q_n+1 = R_n + \overline{S_n} \overline{ Q_n}$
$Q_n+1 = \overline {R_n + \overline{S_n + Q_n}} $\\

Проанализируем его работу.\\

1)R = S = 0\\
а) Q = 0 . \overline{Q} = 1\\
Логический элемент находится в сосотоянии 0, этот 0 приходит на вход логического элемента 2, поддерживая ЛЭ2 в единичном состоянии.
В свою очередь единица на выходы ЛЭ2 подерживает в нулевом состоянии ЛЭ1, то есть в схеме ничего не происходит.

б)
ЛЭ2 находится в состоянии 0, 0 из его выхода подтверждает единичное состояние ЛЭ1.
В свою очередь единица из ЛЭ1 подтверждает нулевое состояние ЛЭ2.\\


2)R = 1; S = 0\\
а) Q = 0;  \overline{Q} = 1 \\
Единица подтверждает нулевое состояние ЛЭ1. Нуль подтверждает единичное состояние ЛЭ2.
Многократное действие сигнала R = 1 подтверждает нулевое состояние.

Триггер превращается в 2 несвязных логических элементов, переставая быть триггером.
Самое страшное произойдет, если одновременно снять сигналы S,R. Мы не знаем в какое состояние перейдет триггер.

\end{document}
