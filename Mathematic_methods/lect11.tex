\LARGE{ \textbf {Лекция №11}}\\
\Large{ \textbf {Дискретное программирование}}\\

\textbf{Предметная область}\\
Дискретное программирование - это раздел математического программирования, который изучает формальные модели
и численные методы поиска оптимально решения на конечном или счетном множестве вариантов. \\

Могут возникать различными путями:
\begin{itemize}
  \item Задачи с неделимостями, где параметрами являются физичиские неделимые велечины.
  \item Задачи комбинаторного типа для оптимальное решение нужно найти в комбинаторном пространстве.
   Комбинаторные объекты сами по себе дискретны.
   Наиболее известные задачи связаны с проблемой выбора(элементов из матрицы). И поиск цепей циклов в конечных графах
   \item При классических задачах математического программирования.
   Это задачи с неоднородной или разрывной функции цели и задачи с алтернативными условиями типа: либо-либо.
   Получается, что эти задачи могут быть сведены к задачам дискретного программирования путем введения дополнительных переменных.
   Это позволяет исключить неклассические особенности функции цели для ограничений.
\end{itemize}

\textbf{Постановка задачи дискретного программирования}\\

В общем случае это постановка требует формального описания управляемых переменных,
 ограничений и функций цели, как принято в математическом програмиировании.
Отличительной особенностью является наличие дескретности или целочисленности управляемых переменных.
Что касается функции цели и ограничений, то это могут быть произвольные линейные или нелинейные функции,
но особенно хорошо изучены задачи с линейными функциями цели и ограничений.
Более того, большинство практических интресеных  задач относятся именно к этому типу.
Условно делятся на две группы: задачи с условием целочисленности переменных ЦЛП, $x_j \in [0,1,2, \cdots]$
И задачи булева программирования БЛП  \\
$x_j \in [0,1] \quad j = 1, \cdots , n$\\
$ \sum \limits_{j =1}^n a_{ij}x_j <> b_i \quad i = 1, \cdots , m$\\
$z = \sum \limits_{j =1}^n c_j x_j - extr $\\

Задачи булева и целочиленного програмиирования могут быть сведены друг к другу.\\
БЛП ту ЦЛП\\
$0 <=x_j <= 1 \quad j = 1, \cdots , n$\\
$x_j  \in [0,1,2, \cdots]$\\

ЦЛП ту БЛП\\
$x_j  \in [0,1,2, \cdots N_j]$\\
$x_j  =  \sum \limits_{k=0}^p  2^k - y_{kj} \quad y_{kj} in [0,1] \quad k = [0, \cdots p_j]$\\
$p_j - ? $\\
$ max p_j : N_j <= 2^{p_{j+1}} - 1$\\
$N_j + 1 <= 2^{p_{j+1}} $\\
$\frac{N_j + 1}{2} <= 2^{p_{j}} $\\
$p_j => log_2 \frac{N_j + 1}{2}$\\

Переход от целочисленного к булеву программированию позволит решать целочисленные задачи путем полного перебора
двоичных кодов значений булевых переменных.
Но в этом случае увеличивается размерность задачи, а для дискретного програмиирования это существенно.
Так как не существует эффективных способов задачи.

Обзор методов дискретного програмиирования.

Методы можно разделить на 2 группы: точные и приближенные.\\
Точные методы гарантируют получение оптимально результата, но их возмоности огрничены размерностью задачи.\\
Приближенные методы могут быть использованы для решения задач любой размерности,
однако не гарантируют полученние оптимально результата, а только приближенное к оптимальному.\\

Точные методы\\
Основыными точными методами являются метод отсечений и метод ветвей и границ.
Метод отсечений основан на том, чтобы построить по исходной целочисленной задачи линейную задачу,
оптимальное решение которой будет совпадать с решением исходной целочисленной.

Метод границ обеспечивает     путем массового отсева неперспективных подмножеств решений, где не может быть оптимальных решений.
