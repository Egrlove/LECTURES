\LARGE{ \textbf {Лекция №4 5.03.18}}\\
\Large{ \textbf {Симплекс-метод}}\\

​
Численный метод линейного программирования, который реализует рациональный перебор базисных решений канонической задачи ЛП в поле конечного итерационного процесса , где на каждом шаге текущий базис уменьшается в направлении оптимального решения.

Каждое следующее решение оказывается, как минимум, не хуже.

Был разработан Джорджем Дантцигом, который впервые применил его для решения задачи системы ограничений, базисные решения которых образуют симплекс многомерного пространства, отсюда название.

Симплекс - выпуклая оболочка n+1 точек n-мерного пространства не все из которых принадлежат одной гиперплоскости.

В настоящее время это наиболее распространённый метод линейного программирования, который позволяет эффективно найти оптимальное решение для большинства практических случаев. При этом число итераций определяется полиномом 3 степени относительно размерности задачи.

Существуют специально подобранные задачи где итерационный процесс зацикливается, не доходя максимум. Значение целевой функции остаётся постоянным, меняются базисы.
Также применяется в целочисленном программировании, алгоритмы используют его как внутреннюю часть.

Существует большое число алгоритмических реализаций, которые связаны одной общей идеей : во всех них вычислительный процесс сводится к последовательности табличных преобразований, основанных на исключении Гаусса для СЛАУ.

Основными симплексными алгоритмами являются: прямой и двойственный алгоритмы симплекс метода. Каждый из них может быть реализован в форме столбцов и строчек симплекс таблиц, что даёт 4 базовых алгоритма.

Прямо-двойственный метод\\
Модифированный метод - больше на вычислительную, если писать программы, то на нем.
Для частных задач: для задач в искусственном базисе. метод больших штрафов ( на лабах).
Метод столбцов больше для целочисленного программирования.
\newpage
\textbf {Прямой алгоритм Симплекс-метода.}\\
\begin{enumerate}
  \item Предварительный шаг: нужно начать с опорного базисного допустимого решения, записать постановку задачи ЛП в канонической форме, которая является диагональной относительно критериальной и базисных переменных. Целевая функция : дописать. (ДИАГОНАЛЬНАЯ ФОРМА ЗАПИСИ)В этой записи постановка задачи линейного программирования представлена в виде системы уравнений, таким образом что каждая базисная переменная входит только одно своё уравнение с коэффициентом единица в остальных уравнения коэффициент перед ней 0. А критериальная переменная z входит в уравнение со знаком плюс или минус .ъъъъ как правило удобно за зависимые взять дополнительные переменные , а за опорные взять базисные.

  \item Нужно представить диагональную форму текущего базиса в форме симплекс таблицы. Ее строки соответствуют уравнениям диагональной матрицы, причём начальная строка соответсвует уравнению целевой функции и может быть помечена критерисльной переменной. Остальные строки соответствуют уравнению ограничений и могут быть помечены их базисными переменными. Столбцы симплекс таблицы соответсвеът всем переменным задачи в порядке их функций , а самой правый столбец нужен для правой части уравнения для канонической формы. в клетках таблицы записываются ——————— Таблица 1.

  \item   Нужно оценить оптимальность текущего базисна начальной строки, критерием оптимальности является неотрицательность всех коэффициентов строки. Если это выполняется нужно перейти на шаг 6.Если же базис можно улучшить, то переходим на след шаг.

  \item Модификация базиса. Производится путём выбора одной из свободных переменных для ввода в базис взамен одна из базисных должна стать свободной. Осуществляется в 2 этапа: сначала нужно выбрать свободную переменную для ввода в базис, это может быть любая переменная с отрицательным коэффициентом в строке Z. Обычно выбирает самый отрицательный. Допустим Aop = min. Столбец выбранной переменной называется ведущим столбцом. Ее значение из нуля станет положительным это приведёт к увеличению значения базисных переменных в ведущем столбце которых стоят отрицательные коэффициенты. Для выбора базисной переменной, которая станет свободной со значением 0, нужно рассматривать отношение правых частей базисных уравнений коэффициенты которых больше 0.Из этих отношений нужно выбрать базисную строку где это значение минимально.Тогда эта переменная раньше других станет свободной - обратится в 0. Ее строка - ведущая строка. Элемент на пересечении строки и столбца называется ведущим. Таблица перестаёт быть диагональной в новом базисе.
  \item Преобразование таблицы к диагональной форме относительно нового базиса. Оно основано на процедурах исключения Гаусса для СЛАУ цель этих исключений : сделать ведущий элемент равным 1 и обнулить остальные коэффициенты ведущего столбца.Такое преобразование делается в 2 этапа, а результаты записываются в новую таблицу по предыдущей. Шапка коэффициентов не нужна, а обозначение строк меняется. Остальное никак не меняется.Упорядочивание строк будет считаться ошибкой. Сначала ведущий элемент делаем равным единице: разделить на этот коэффициент . Чтобы обнулить ведущий столбец нужно выполнить следующее со всеми строками, кроме : . Сделать нужно. После этого вернуться на шаг 2 для прорисовки новой таблицы. Такие итерации выполняются до тех пор, пока не будет выполнено условие оптимальности из пункта 3.
\item В результате оптимальные значения базисных переменных берутся из правого столбца таблицы, оптимальное значение целевой функции берётся из первого столбца таблицы. Свободные переменные должны быть равны 0.

\end{enumerate}
