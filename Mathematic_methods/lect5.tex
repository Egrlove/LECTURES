\LARGE{ \textbf {Лекция №5}}\\
\Large{ \textbf {Пример прямого алгоритма симплекс-метода }}\\
.\\
$
z = x_1 + 4x_2 \mbox{ - max}$ \\
$x_1 + x_2 <= 4 $ \\
$-x_1 + x_2 <= 2 $ \\
$2x_1 + x_2 <= 6 $ \\
$x_1 \cdot x_2 => 0 $ \\

------------------------------\\
$
z - x_1 - 4x_2 = 0 \\
x_1 + x_2 + x_3 = 4 \\
-x_1 + x_2 +x_4 = 2\\
2x_1 + x_2+ x_5 = 6 \\
x_1x_2x_3x_4x_5 => 0 \\
x_3,x_4,x_5 \mbox{- Начальный опорный базис}\\
$
------------------------------\\
Таблица 12 строк, столбцов -  все переменные плюс правые части \\
В базис принято вводить минимальное отрицательное, то есть $x_2 = -4$ \\ Помечаем точкой.
Для вывода рассматриваются только положительные переменные. \\
Быстрее всех обратиться в ноль та переменная, где значение отношения минимально. $x_4$ Помечаем *.\\
Нужно привести к диагональной форме. Используя первое исключение Гаусса:\\
Делим на значение *, всю строку замененой переменной. \\
Второе Исключение: -4 умножаем на всю ведущую строку, затем вычетаем получившееся из строки целевой функции Z.\\
Из строки $x_3$ вычитается $x_2$. \\

Нужно новую ведущую строку х1 умножить на коэфицент строки Z ведущего столбца -5  и вычесть из строки Z. \\
Смотри на строку Z, если все кэфы положительны значит найдено оптимальное решение. \\
Пишем ответ: \\
$
z = 13 \\
(x_1)^* = 1 \\
(x_2)^* = 3 \\
(x_5)^* = 1 \\
(x_3)^* = (x_4)^* = 0 \\
$


\Large{ \textbf {Анализ прямого алгоритма симплекс-метода }}\\
Алтернативные решения.\\
Под ними понимается бесконечное множество оптимальных решений, которые равны по значению целевым функциям,
 но отличаются по значениям переменных. В некоторых задачах могут появляться такие решения.\\
Геометрическая интерпретация\\
Линии равного уровня буду параллельны активной границе ОДР
(Уравнение которое обраещается в тождество, при подстановке в него оптимального решения.).

\Large{ \textbf { Как бесконечно много решений отображается на таблице ?}}\\
Для итераций формальным признаком наличия оптимального решения является равенство 0.
Коэфицента в строке Z перед одной из свободных переменных в оптимальной симплекс-таблице.
В этом случае ее значение можно варьировать в положительном интервале.

Свойство выпуклости:
Любая линейная комбинация для двух произвольных векторов в множестве дает новое решение. \\
$
X^* = \lambda X^A + (1 - \lambda)X^B \\
x_1 ^ * = \lambda x_1 ^A + (1 - \lambda)X^B_1 \\
x_2 ^ * = \lambda x_2 ^A + (1 - \lambda)X^B_2 \\
$
