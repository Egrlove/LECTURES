\LARGE{ \textbf {Лекция №3}}\\
\Large{ \textbf {Многопоточное программирование}}\\

Недостатки многопроцессорного программирования:\\
Каждый процесс требует прикладных расходов. (Диспетчиризация процесса и тд).\\
Передача сложных данных через канал не очень, но затруднительно. (Будет проще использовать разделяемую память).

Распаралеливание в рамках одного процесса, работа ведется в рамках одного процесса (в одной области памяти)


$int pthread_create( pthread_t *tidp_, pthread_attr_t *att2_p, void*(func_p)(void *), void* arg_p)$\\
$attr_p$ - атрибуты потока.\\
$pthread_attr_init(_att2_t * att2_p)\\$
Атрибуты: \\
\begin{enumerate}
  \item Область действия конкуренции scope/
  \item состояние присоединения / отсоединения . Есть возможность ожидать
  \item адрес стека
  \item размер стека
  \item Приоритет потока (По умолчанию наследуется от родителя)
  \item Политика диспечеризирования потоков( Можем поппытаться сказать, как планировать потоки на выполнение)
\end{enumerate}
$pthread_attr_destroy(_att2_t * att2_p)$ \\
Одна структура может использоваться при создании многих потоков.\\
Работа с область конкуренции
$int pthread_attr_setscope(int pthread_att int scope) $\\
PTHREAD_SCOPE_PROCESS - БЛАСТЬ КОНКУРЕНЦИИ  один процесс \\
PTHREAD_SCOPE_SYSTEM - ОБЛАСТЬ ВСЯ СИСТЕМА.\\
$int pthread_att_getscope(int pthread_att ,int * pscope)\\$
Атрибуты для состояния определенности:\\
int pthread_attr_setdetachstate( pthread_attr_t * attr_p, int state) . state может принимать 2 значения\\
  PTHREAD_CREATE_DETACHED  - НЕТ ВОЗМОЖНОСТИ ОЖИДАТЬ\\
  PTHREAD_CREATE_JOINABLE - можно не только ожидать, но и получать код возврата.\\
int pthread_attr_getdetachstate( pthread_attr_t * attr_p, int* state) . state может принимать 2 значения\\
pthread_exit(void * status) - явно завершает поток, при условии, что поток создан как joinable. \\
int pthread_join(pthread_t tid(завершение которого ожидает), void **status(те значения которые выдал поток)) \\
Можно узнать собственный идентификатор: \\
pthread_t self(); \\

Потоки могут синхронизироваться путем посылки сигналов:
int pthread_kill(pthread_t tid, int signum) \\
Кому шлется, номер сигнала.\\

Механизмы синхронизации:\\
\begin{itemize}
  \item mutex - взаимоисключающие блокировки
  \item Условные переменные
  \item Семафоры
  \item Барьер
\end{itemize}

mutex\\
Может быть только один владелец, при поптытках других они будут блокироваться.
После разблокировки право получает тот, у кого выше приоритет.

int pthread_mutex_init( pthread_mutex_t *mp, pthread_mutex_attr_t * mattr_p)\\
Указатель на mutex, структура(можно задавать NULL, так и будем делать)\\
После того, как mutex не нужен:\\
int pthread_mutex_destroy(pthread_mutex_t *mp)\\
int pthread_mutex_lock() - поптыка захватить \\
int pthread_mutex_unlock() - освобождение, только хозяин. \\
int pthread_mutex_trylock() - можно проверить. \\


Условные переменные\\
Используются вместе с mutex\\
Захватывает одип поток. \\
Освобождает на время, по условной переменной осуществояется передача.\\

int pthread_cond_init (pthread_cond_t *cvp ,pthread_cond_attr_t *cattr_p)\\
int pthread_cond_destroy (pthread_cond_t *cvp) - указатель на переменную. \\
int pthread_cond_wait(pthread_cond_t *cvp , pthread_mutex_t *mp) - постоянный владелец по переменной освобождает этот mutex \\
int pthread_cond_signal(pthread_cond_t *cvp) - передача сигнала, о завершении временного освобождения \\

\Large{ \textbf { Семафоры. }}\\
Для использования нужно:\\
semaphore.h\\
Переменная с которой возможны только два действия : инкремент и декримент.\\
Если поток пытается сделать меньше нуля, то он будет блокироваться.\\

int sem_init(sem_t *sp, int pshared, u_int value)
область действия семафора (0 - один единственный процесс, не нуль - на несколько процессов) ,начальноное значение.\\
int sem_init(sem_t *sp)\\
int sem_post(sem_t *sp) - увеличение на 1. \\
int sem_wait(sem_t *sp) - попытка уменьшить значние семафора на единицу.\\
int sem_trywait(sem_t *sp) - неблокирующая версия.\\

\Large{ \textbf { Барьер. }}\\

Некоторая точка, продолжение после которой будет только если потоков перед ним наберется равное ширине барьера.\\
int pthread_barrier_init(pthread_barrier_t *bp, NULL, int width)\\
int pthread_barrier(pthread_barrier_t *bp)\\

\Large{ \textbf { Код примера. }}\\
Конвеерная обработка данных, есть три модуля:\\
1 производит - producer ( строка )\\
2 обрабатывает - invertor \\
3 получатель - consumer\\
Для передачи строки будем использовать буффер msgbuf1 1-2 , msgbuf2 2-3 \\
msglen1, msglen2 - длина строки.\\
mtx1, mtx2 - mutex \\
cv1, cv2 - условные переменные.\\

Поставщик данных завхватывает mutex, инвертор пытается захватить его, он блокироуется,
засыпает проц время не жрет. Спит пока поставщик не освободит перменную.
