\documentclass{report}
\usepackage[left=2cm,right=2cm,top=2cm,bottom=2cm,bindingoffset=0cm]{geometry}
\usepackage[utf8x]{inputenc}
\usepackage[english,russian]{babel}
\usepackage{amsfonts}
\usepackage{amsmath}
\usepackage{mathtools}

\usepackage{graphicx}
\graphicspath{{pictures/}}
\DeclareGraphicsExtensions{.pdf,.png,.jpg}

\begin{document}
\LARGE{ \textbf {Лекция №12}}\\
\Large{ \textbf {Алгоритм Лэнд и Дойг}}\\
Это алгоритмическая реализация метода ветвей и границ для решения задачи ЦЛП.\\
В этом алгоритме исходная целочисленная задача заменяется списком ее ослабленных линейных подзадач,
где условие целочисленности заменено на условие неотрицатеьности.
Подзадачи списка иерархически упорядоченны по своим допустимым решениям ,
это означает что области допустимых решений  подзадач являются подобластями области исходной задачи.
В соответсвии с общей схемой метода ветвей и границ список эти подзадач рекурсивно формируется  и расссматривается
в процедуре ветвления и оценке границ.

\textbf{Процедура ветвления }\\
Пусть на очередной итерации ис списка подзадач выбрана наиболее перспективная из подзадач,
однако не является целочисленным по одной переменной, $x_i$ - эту перменную можно выразить через отрицание свобоных переменных,
как принято в столбцовом Симплекс-методе.
$x_i = A_{i0} + \sum \limits_j A_{ij}(-x_j^(i=0)$\\

Чтобы значение $x_i$ стало целым оно должно быть либо уменьшено до ближайшего целого снизу, дибо увеличено до ближайшего целого сверху.
Можно подставить в эти дополнительные условия выражения для базисной переменной $x_i$.
Неравенство в левой подзадаче $x_i <= [A_{i0}]$\\
Неравенство в правой подзадаче $x_i => [A_{i0}] + 1$\\

В табличной форме, чтобы получить 2 новые подзадачи в списке нужно взять 2 копии оптимальной симплекс
таблицы выбранной перспективной от задачи списка, каждой копии нужно снизу добавить по одной строке для слабой переменной $y_i$
В левой подзадаче свободный член этой строки равен отрицанию дробной части свободного члена образующей строкию,
а все коэффиценты равны отрицанию коэффицентов образующей строки $x_i$
В правой подзадаче свободный член дополнительной сроки равен дробной части свобоного члена образующей строки $x_i - 1$ ,
а все коэффиценты дополнительной строки равны кофэффицентам образующей строки.
\textbf{Процедура оценки границ.}\\
Это процедура должна осуществлять анализ перспективности подзадач списка - эта оценка определяется
При поиске максимума перспективной считается задача с максимальным оптимальным значением Z. При поиске минимума - минимально.
Для решения подзадач списка используется столбцовый алгоритм симплекс-метода, при этом начальная исходная ослабленная задача
может решать либо прямым, ибо двойственным симплекс методом, в зависимомти от того является ли её опорный базис прямо или двойственнодопустимым.
А каждая ис последующих задач решается столбцовым методом, при этом на первой симплексной итерации
за ведущую строку всегда выбирается дополнительная строка, для слабой переменной y.
Потому что это единственная отрицательная переменная в строке.
Процедуры ветвления и оценки раниц чередуются на итерациях алгоритма.
А паралельно с решением задачи строится дерево решения. Его узлы - соответсвую подзадачм из списка,
ветви маркируются дополнительными условиями по дробной переменной и отражают иерархию подзадачт.
Решение должно продолжаться, пока не будет получена подзадача с оптимальным целочисленным решением,
которое является лучшим для подзадач в списке.
\textbf{Пример алгоритма.}\\

Найти максимальное значание функции цели \\
$z= x_1 + 2 x_2$\\
$2x_1 + 2x_2 <= 7$\\
$4x_1 - 5x_2 <= 9$\\
$x_1 , x_2 \in [0,1,2..]$\\

Canon\\
$z = x_1 + 2 x_2 $ \\
$2x_1 + 2x_2 + x_3^B = 7$\\
$4x_1 - 5x_2 + x_4^B = 9$\\

Dicanon\\
$z =  -(-x_1) -  2 (-x_2)$\\
$x_3 = 7 +2(-x_1)  + 2 (-x_2)$\\
$x_4 = 9 +4(-x_1)  + 5 (-x_2)$\\

$ z_0 = 7 $\\
$x_1=0$\\
$x_2 =0  $\\

В ходе ветвления их значения не возрастают (оценок).


\end{document}
