\LARGE{ \textbf {Лекция №13}}\\

\Large{ \textbf {Метод сечений}}\\

Это алгоритмическая реализация метода сечения, где используеется дробное соотношение коэффицентов для исключения дробных результатов.
Основная идея следующая: рассматривается ослабленная задача нужно заменить на условие неотрицательности переменных.
Это задача решается как задача симплекс-методом.
Если получается дробное оптимальное решение хоть по одной перменной нужно рассширить систему ограничений дополнительным условием.
Которое называется отсечением.
Обладает следующими свойствами:
\begin{enumerate}
  \item Полученный дробный резултат ему не удовлетворяет.
  \item Любое целочисленное допустимое решение ему удовлетворяет.
\end{enumerate}

После этого рассматривается рассширенная ослабленная задача с этим дополнительным условием, которая обрабаытвается аналогичным образом.
Этот процесс нужно продолжать до тех пор пока при решении очередной задачи не будет получен оптимальное целочисленное решение.
Фактически аппроксимирует дробный к ближайшему целому результату.

Для реализации был предложен ряд алгоритмов, которые формализуют процедуру построения правильных отсечений.
Один из них - "Дробный алгоритм". (Кроме того существует дискретный алгоритм,
а так же смешанный(Применяется для решения частично целочисленных задач.))

На каждой итерации для решения ослабленной задачи применяется столбцовый алгоритм симплекс-метода.
В этом случае для выполнения его шагов нужно предствать ослабленную задачу в канонической форме, где критериальная и базисные перменные
выражены через отрицание свободных переменных, а свободные через отрицание самих себя.\\
\\
$x_i = A_{i0} + \sum \limits_j A_{ij}(-x_j); \quad i = 1,....,m $\\

В зависмиомти от того, является опорный базис прямо или двойственно допустимым. Только на начальны итерациях.
На последующихх итерациях используется только двойственный вариант. По тем же причинам, что в методе ветвей и границ.
Если её решение получилось целочисленным, то оно считается оптимальным решением исходной задачи и процесс вычисления закончен.
Если оно получилось дробным, тогда нужно в оптимальной таблице выбрать производящую строку:
это может быть любая строка  с дробным значаением свободным членом.\\
$x_q = A_{q0} + \sum \limits_j A_{qj}(-x_j); \quad X_q = A_{qj} \neq 0 \quad mod \quad 1 $\\
$A_{q0} = [A_{q0}] + \{ A_{qj} \}  \quad  0<A_{qj}<0 \quad 0< A_{q0} <1 $\\
$A = - 5/4 \quad [A=-2] \quad \{A = 3/4 \} $ \\

По кофэффицентам уравнения дробного отсечения строится дополнительная строка отсечения, которая дополняется к оптимльной таблице снизу.
Это расширенная таблица считается начальной следующей большой итерации дробного алгоритма.
Её обработка осуществляется двойственным столбцовым методом.
А сама строка будет ведущей по тем же причинам, как в алгоритме ветвей.
Симплексная итерациявыполняется двойственным образом до тех пор, пока таблица не станет двойственно допустимой.
Рассмотренный процесс сопровождается увелечением числа строк, есть возможно сохранить размеры таблицы на всех итерациях.
После первой симплексной итерации в очередной большой итерации ведущая дополнительная строка становится свободной,
а слабые переменные становится свбодными
Перед тем, как нужно исключть эту свободную строку из таблицы, тогда число строк всегда будет равно.
Но в этом случае может уеличиться колво итераций.



\Large{ \textbf {Дискретный алгоритм}}\\

Алгоритмическая реализация метода сечения, которая гарантирует целочисленный характер симплексных преобразований на всех своих итерациях.
Он был разработан, как алтернатива дробному алгоритму, чтобы исключть для удобства обработки дробных велечин.

Для целочисленности всех сеимплекс таблиц требуется , чтобы ведущие элементы на всех итерациях были равны -1.ы

Для достижения такого эффекта нужно изменить принцип построения отсекающих строк.

Дискретный алгоритм ориентирован на обработку ослабленных задач с двойственнодопустимым опорным базисом.
Это задача поиска минимума с неотрицательными кофэффицентам - можно решать, другие требуют специальныъ мер.

Схема:
\begin{enumerate}
  \item Сначала, как обычно от целочисленной переходим к ослабленной,
  заменив условие целочисленности на условие неотрицательности перменных.
  Затем написать постановку задачи в канонической форме, где критериальная и базисные перменны выражены через отрицание свободных перменных.
  а свбодные через отрицание самих себя.
  Дальше по этой канонической форме строится начальная симплекс таблица, для её обработки выполняется последовательность итераций и следующих шагов.
  \begin{enumerate}
    \item Выбор производящяй строки. Можно за неё выбрать любую строку с отрицательным значаением свободного члена. Обычно выбирается строка с минимальным значением.\\
    $x_q = A_{q0} = min(A_{i0}) < 0 $\\
    \item Выбор столбца. Это должен быть лексиграфически мимнимальный столбец, из тех который имеют отрицательный коэффицент в производящей строке.

    \item Для кажого столбца с отрицательным  нужно вычислить нормировочный коэффицент по след формуле.\\
    $u_j = \frac{A_{0j}}{A_{0p}} \quad при  A_{qj} < 0 \quad j>0$ \\
    Нормировочный коэффицент для направляющего столбца всегда равен 1.
    Но может быть так, что в напр столбце кэф = 0, тогда нормировочный кэф для других стобцов будут равны бесконечности. Но для направляющего все равно 1.
    \item Нормировка проивзодящей строки. Для каждого отрицательного коэффицента производящей строки нужно вычислить нормировочный коэффицент по формуле:\\
    $v_j = \frac{-A_{qj} < 0}{u_j \geq 0}  \geq 0  \quad при   \quad j>0$\\
    $X_p = \frac{- A_{qp} }{u_p =1 }$\\
    Теперь из всех нормировочных кэфов строки нужно выбрать максимум. $v$
    \item Построение сечения. строится из норм кэфов строки и\\
    $ \frac{A_{q0}}{v>0} => \sum \limits_j  \frac{A_{qj}}{v>0}  $\\
    Дальше по неравенству строится уравнение дискретного отсечения.\\
    $ y_q = \frac{A_{q0}}{v} +  \sum \limits_j  \frac{A_{qj}}{v} \cdot (-x_j)   $

    Важно что кэф в направляющем столбце будет равен -1.
    \item Строку с кэфами уравнения дискретного отсечения надо приписать снизу к таблице. После этого выбрать доп строку в качестве ведущей, а за ведующий столбце выбрать направлющий столбец p.
    Выполнить одну итерацию столбцово двойственного симплекс-метода, если после этого в столбце свободных членов получатся неотрицательными, то это решение теперь и прямо и двойственно допустимо, то есть оптимально.
    Принимается  за результат задачи.
    Если после итерации будут отрицательные нужно перейти к следующей итерации с тем же набором шагов, который был рассмотрен.
  \end{enumerate}
\end{enumerate}
