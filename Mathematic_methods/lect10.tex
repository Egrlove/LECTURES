\documentclass{report}
\usepackage[left=2cm,right=2cm,top=2cm,bottom=2cm,bindingoffset=0cm]{geometry}
\usepackage[utf8x]{inputenc}
\usepackage[english,russian]{babel}
\usepackage{amsfonts}
\usepackage{amsmath}
\usepackage{mathtools}

\usepackage{graphicx}
\graphicspath{{pictures/}}
\DeclareGraphicsExtensions{.pdf,.png,.jpg}

\begin{document}
\LARGE{ \textbf {Лекция №10}}\\
\Large{ \textbf {}}\\
Алгоритмическая реализация симплекс-метода,
где исключение перменных после модификации базиса производится по столбцам таблицы вместо строк, как в классическом варианте.
При этом сама симплекс таблица имеет столбцовый формат, вместо классического строчного.
Чтобы построить стобцовую симплекс таблицу нужно из уравнений какнонической формы задачи
явно выразить базисную и критериальную переменные, через отрицание свободных перменных.
А свободные переменные нужно выразить через отрицание самих себя. В результате этих преобразований получается
следующая система уравнений: \\

$x_0$ - критериальная\\

$x_0 = A_{00} + \sum \limits^n_{j = 1} A_{0j}(-x_j^F) $ \\
$x_{n+1}^B = A_{i0} + \sum \limits^n_{j = 1} A_{ij}(-x_j^F) $ \\
$ x_j^F = (-1) (-x^F_j ) j = 1,...,n $ \\

По этой канонической форме строится столбцовая симплекс таблица,
которая имеет следующие поля:\\
\begin{itemize}
  \item строки соответсвуют уравнениям какнонической форме и каждая строка маркируется переменной левой части соответсвующего уравнения.
  \item столбцы этой таблицы обозначаются отрицанием свободных переменных
  \item кроме того имеется специальный столбец, который обозначен 1 для свободных членов этих канонических уравнений.
  \item В клетках таблицы записываются коэфиценты уравнений канонической формы.\\
\end{itemize}


Таблица диаганальной формы относительно свободных переменных.\\
Обработка столбцовых симплекс таблиц может быть реализована, как прямым так и двойственным сиплекс методом,
в зависмомти от того какой опорный базис у задачи.
Если двойственно допустимый , то двойственный симплекс метод.\\
Если прямодопустимый, то прямой.\\

Последовательность симплекс операция для обработки столбцовых таблиц такая же как для обработки строковых таблиц.
Отличие только на шаге исключения Гаусса, которое выполняется после модификации базиса ,
для приведения таблицы к диагональной форме относительно свободных переменных ( в этом варианте)
При это исключение Гаусса производится по стобцам таблицы, вместо по строкам.

Пусть $X_p$ свободная переменная вводится в базиз, вместо неё $X_q$ должна стать свободной , текущий элемент помечен звездочкой.\\
Тогда после исключения Гаусса строка должна стать свободной строкой(строкой дл) она будет состоять из нулей,
кроме того , что ведующий элемент в ней будет равен $-1$.
Как обычно исключение производится в 2 этапа.\\
Первый этап - обработка ведущего столбца - его цель сделать ведущий элемент равным -1.
Для этого нужну ведущий столбец разделить на минус ведущий элемент.

$A`_{ip} =  - \frac{A_{ip}}{A_{qp}}$\\

$A`_{qp} =  - \frac{A_{qp}}{A_{qp}} = -1$\\

Второе исключение выполняется для всех столбцов таблицы кроме ведущего,
его целью является обнулить все коэфиценты ведущей строки, кроме ведущего элемента.

Формула преобразования :\\
Допустим j под обработкой.\\
$A`_{ij} = A_{ij} + A_{qj} \cdot A`_{ip} , j != p  $\\

После того как произведено исключение Гаусса строится новая таблица. Обозначение строк в ней не меняется,
а изменеяется только обозначение ведущего столбца.\\
Если таблиц не много то их можно рисовать рядом друг с другом слева-направо. \\

Строковый алгоритм обычно используется в методах целочисленного, как составная часть.\\
$z  = 3x_1  + x_2 ----min $\\
$3x_1 + 5x_2 => 15 $ \\
$5x_1 + 3x_2 =>15 $\\
Canon\\
$ -z = -3x_1 - x_2 -----max $\\
$3x_1 + 5x_2 - x_3^B = 15 $\\
$5x_1 + 3x_2 -x_4^B = 15  $ \\
-Canon \\
$ -z = 3 (-x_1)  +(-x_2)  $\\
$ x_3^B = -15  - 3(-x_1) - 5(-x_2) $\\
$ x_4^B = -15 - 5(-x_1) - 3(-x_2)$\\

$x_1 = (-1)(-x_1) $\\
$x_2 = (-1)(-x_2) $\\


\end{document}
